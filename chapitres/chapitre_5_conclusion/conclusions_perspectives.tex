\chapter{Bilan et perspectives de travail}\label{chap:conclusion-persp}

\vspace{1cm}

\minitoc

\newpage

\section{Apport global}

Nous avons consacré ce manuscrit de thèse à l'étude numérique du comportement de problèmes multi-contacts aussi bien en milieu déformable qu'en milieu granulaire. Ainsi, après avoir introduit le cadre physique et numérique dans la première partie, nous nous sommes intéressés dans le premier chapitre de la seconde partie à l'analyse mathématique et la résolution numérique de trois différents problèmes élastiques et hyper-élastiques de contact sur une fondation rigide, telles que rapportées dans l'article paru \cite{abide2021inexact}, à savoir le contact unilatéral, le contact bilatéral avec frottement de Tresca et le contact unilatéral avec frottement de Coulomb. Pour chacun d'entre eux, nous avons fourni une analyse particulièrement détaillée des différentes méthodes de résolution type PDAS basées sur la reformulation des conditions de contact et de frottement sous formes de fonctions de complémentarité non-linéaires, et dont la solution est donnée par la méthode itérative semi-régulière de Newton. Nous avons pu mettre en lumière, à travers une série de simulations numériques dans le cadre statique puis dynamique, la pertinence et les performances de ces méthodes comparé à la méthode du quasi-Lagrangien Augmenté dans la résolution de problèmes issus de la mécanique du contact en petites et grandes déformations. Nous retiendrons plus précisément à l'issue des simulations menées que le nombre d'itérations des méthodes PDAS est légèrement plus important que le quasi-Lagrangien augmenté, mais qu'en terme de temps de calcul, les méthodes PDAS sont nettement moins coûteuses en raison de la taille du système non-linéaire à résoudre et surtout de la symétrie des systèmes linéaires dû à l'absence des multiplicateurs de Lagrange. De surcroît, la méthode PDAS semble être adaptée pour des configurations dynamiques complexes présentant un grand nombre de noeuds de contact, puisqu'il a été observé qu'on avait des temps CPU qui étaient inférieurs  en comparaison avec le quasi-Lagrangien augmenté, et que l'écart de temps CPU s'élargissait par rapport à des simulations moins complexes. Enfin, nous avons pu mettre en avant la robustesse des méthodes type PDAS pour la résolution numérique de problèmes de contact complexes en milieu continu en l'adaptant à la configuration de l'anneau hyper-élastique.\\

Dans le cadre de la simulation numérique des milieux granulaires, nous avons présenté dans le chapitre $3$, sur la base des travaux présentés dans les articles \cite{abide2021semismooth} et \cite{abide2021unified}, respectivement accepté et soumis, une généralisation de l'approche NSCD-PDAS pour la résolution numérique des lois de contact et de frottement, sans avoir recours à un processus de régularisation. Ainsi, l'approche PDAS a été étendue aux problèmes multi-corps rigides, en détaillant le traitement numérique des conditions de contact avec frottement par la méthode itérative semi-régulière de Newton et en les explicitant sur chaque sous-ensemble en termes d’impulsions de contact, ce qui nous a conduit à l'expression de deux algorithmes de type PDAS exact et itératif. De plus, un soin particulier a été attaché à la mise au point d'un algorithme de général de résolution. Dans le reste du chapitre, nous avons été en mesure de mettre valeur la pertinence et les performances des méthodes types PDAS dans le cadre de la dynamique multi-corps rigide, puisqu'un ensemble d'expérimentation numérique est conduit pour valider cette nouvelle approche. Ainsi, il apparaît clairement au regard de toutes ces expérimentations que les méthodes PDAS donnent d'une part des résultats numériques précis et similaires aux solutions analytiques sur des configurations élémentaires, et sont d'autres part plus efficaces en termes de convergence numérique et de temps de calcul comparées à la méthode standard du quasi-Lagrangien augmenté et du bi-potentiel amélioré, notamment sur les configurations de sédimentation de particules et d'écoulement dans un tambour. Cependant, un constat que l'on est en droit d'attendre se dégage. En effet, dès lors qu'il s'agit de traiter des simulations représentatives des milieux granulaires caractérisés par une multitude de corps rigides et un grand nombre de contacts simultanés, les temps de calcul deviennent très grands, bien qu'ils restent moins importants par rapport aux autres méthodes numériques.\\

Enfin, nous avons présenté dans le chapitre $4$ de ce manuscrit la mise en oeuvre de l'approche numérique NCSD-PDAS dans un contexte applicatif. On s'est donc proposé d'implémenter l'approche dans le solveur open-source MFiX-EXA. Ce logiciel, dédié à la simulation des écoulements multiphasiques fluide-particules dans un environnement massivement parallèle, nous
a permis d’hériter du parallélisme, incontournable pour ce type d'applications, et de valider l'approche NSCD-PDAS pour les écoulements multiphasiques. Dans un premier temps, nous avons introduit ce chapitre par une description du modèle de contact régulier fournit par l'approche DEM-CUNDALL retenue par les développeurs de MFiX-EXA, en soulignant les contraintes portant sur la stabilité du schéma temporel. Cela nous a permis d'enchaîner sur une description des différents éléments algorithmiques du solveur multiphasique nécessaires à l'implémentation de l'approche NSCD-PDAS, en mettant en avant le paradigme de parallélisation proposé ainsi que la modélisation physique de l'interaction fluide-particules. Partant de là, nous avons adopté la technique de parallélisation basée sur la décomposition de domaine géométrique, puis l'avons adapté à notre formalisme numérique NSCD-PDAS, en héritant de la structure algorithmique du solveur.\\
Après une présentation de l'implémentation de la NSCD-PDAS dans l'environnement MFiX-EXA, nous avons entrepris dans la seconde partie du chapitre quelques études comparatives entre les approches DEM-CUNDALL et NSCD-PDAS. Les éléments de comparaisons, qui ont principalement porté sur les aspects d'efficacité informatique incluant la parallélisation ainsi que l'impact de la régularisation DEM-CUNDALL sur la solution numérique finale, ont révélé plusieurs points; d'abord sur les configurations mono-particules présentant une solution analytique, une concordance des résultats numériques obtenus par les deux approches, et une stabilité numérique temporelle supérieure de l'approche NSCD-PDAS, contrairement à l'approche DEM-CUNDALL où les pas de temps deviennent de plus en plus petits lorsqu'un contact rigide est considéré. En ce qui concerne les simulations d'écoulement granulaires purs, l'étude comparative entre les deux approches sur un tambour rotatif a montré que les profils d'écoulement en régime établi étaient similaires. Mieux encore, la condition de persistance formulée dans le cadre de la NSCD assure à la fois une non-interpénétrabilité et une conservation de la quantité d'énergie en régime établi, en comparaison avec l'approche DEM-CUNDALL. D'un point de vue performance informatique, nous avons exploité le parallélisme proposé par le solveur pour obtenir des performances identiques en terme de scalabilité comparé à l'approche DEM-CUNDALL, bien que le paradigme de décomposition de domaine n'offre pas un bon équilibrage de la charge de calcul sur des configurations d'écoulements granulaires dans un tambour rotatif. Enfin, dans le registre des écoulements granulaires couplés au fluide, et contrairement à l'approche DEM-CUNDALL dont les fortes valeurs de la régularisation se traduisent par des pas de temps faibles, nous retiendrons le caractère implicite de l'approche NSCD-PDAS que nous avons pu mettre en avant à travers la simulation du lit fluidisé de Goldschmidt et qui nous permet de relaxer les pas de temps solides et par conséquent une réduction des temps de calcul significative.

\section{Perspectives ouvertes et pistes de réflexion}

Au vu des résultats que nous avons obtenus, plusieurs perspectives et pistes de réflexion ouvertes par cette thèse peuvent être envisagées. Ainsi, il conviendrait d'une part d'envisager l'extension de ces travaux au 3D afin de couvrir un large éventail d'écoulements granulaires aux comportements complexes. La gestion du contact au moyen d'une méthode de type Active Set en 3D et en parallèle reste, à notre connaissance, une question ouverte qui mérite d'être étudiée.\\

Par ailleurs, dans la continuité du travail d'analyse proposé dans le chapitre 4, il pourrait être intéressant de considérer dans le registre de l'interaction fluide-structure d'autres problématiques, principalement les lits fluidisés ou encore les mouvements de foules, afin de valider la pertinence de l'approche NSCD-PDAS conservatrice et les performances liées à son caractère implicite. D'ailleurs, cet aspect peut être envisagé plus globalement dans la perspective d’une parallélisation plus aboutie de l'approche NSCD-PDAS dans le solveur multiphasique MFiX-EXA, ce qui permettrait de réaliser d'une part des gains de temps de calculs significatifs, et de traiter d'autre part des problèmes
d’écoulements non compressibles complexes, tels que proposés dans la plateforme numérique parallèle Xper \cite{dbouk2016df}. Les avancées récentes des méthodes à frontières immergés laissent à penser que l'on est en capacité de simuler des écoulements autour des particules. Dans cette perspective, il peut être intéressant de mettre en place des études comparatives entre Xper et MFiX sur des écoulements tridimensionnels turbulents chargés en particule.\\

Enfin, il serait envisageable d'étendre l'approche DEM-CUNDALL en milieu granulaire, au regard des limitations dues à son caractère régulier, à une DEM améliorée qui préserverait la conservation de l'énergie. On peut d'ailleurs citer les travaux de Hauret et Le Tallec \cite{hauret2006energy} qui introduisent des méthodes d'intégration temporelle permettant de contrôler l’énergie sur des systèmes élastodynamiques en contact via une pénalisation normale à impact modéré. Une étude paramétrique en fonction du coefficient de pénalisation et surtout du pas de temps serait indispensable pour analyser l'évolution de l'énergie discrète des systèmes granulaires.