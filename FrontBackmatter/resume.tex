%*******************************************************
% Résumé
%*******************************************************
%\thispagestyle{empty}
%\phantomsection 
\refstepcounter{dummy}
%\cleardoublepage
\pdfbookmark[1]{Résumé}{Résumé}
\chapter*{Résumé}


De nos jours, l'appréhension des milieux granulaires et de leurs comportements devient un enjeu majeur pour les industriels de plusieurs secteurs d'activité. La communauté scientifique qui s'intéresse particulièrement à l'étude de ces milieux se trouve non seulement confrontée au caractère complexe des interactions régissant leur dynamique, mais surtout au besoin récurrent et croissant de les simuler. Face à ces défis, des outils adaptés permettant de modéliser la dynamique granulaire ont vu le jour, principalement regroupés sous le terme de Méthode des Éléments Discrets (DEM). La dynamique granulaire est alors gouvernée par la deuxième loi de mouvement  de Newton combinée à un  modèle  de contact régulier. Plutard, d'autres stratégies de simulations par éléments discrets ont été élaborées, notamment l'approche \textit{Non-Smooth Contact Dynamics} qui consiste à prendre en compte les interactions de contacts avec frottement collectivement au cours d'un pas de temps, sans avoir recours à un processus de régularisation, et dont La gestion des équations de la dynamique non-régulière se fait au travers de l'algorithme de Gauss-Seidel non-linéaire (NLGS). La littérature regorge de méthodes numériques permettant de résoudre ces équations, mais récemment, les méthodes type Primal-Dual Active Set (PDAS) ont émergé comme des méthodes simples à mettre en oeuvre et efficaces pour résoudre ce genre de problèmes. Ces méthodes sont basées sur le principe suivant: les conditions de contact et de frottement sont reformulées sous forme de fonctions de complémentarité non-linéaires dont la solution est fournie par la méthode itérative semi-régulière de Newton. Sur la base de ces prérequis, l'objectif de ce travail de thèse vise à fournir une généralisation de l'approche non-régulière NSCD-PDAS aussi bien pour les problèmes de contacts élastiques et hyperélastiques que pour ceux en dynamique multi-corps rigide. Un soin particulier est attaché à la mise au point d'un algorithme de résolution des lois de contact et de frottement dans le cadre non-régulier de la NSCD en milieu granulaire. Plusieurs simulations numériques sont rapportées à des fins de vérification et de validation, mais également pour évaluer l'efficacité et les performances des méthodes PDAS par rapport à d'autres méthodes numériques. Enfin, dans un soucis d'intégration de l'approche NSCD-PDAS dans un contexte purement applicatif, l'implémentation a été réalisée dans un solveur open source permettant à la fois d'hériter du parallélisme massif, de simuler des écoulements granulaires couplés au fluide, et de comparer les performances par rapport à la DEM propre au solveur.  


\textbf{Mots-clés:} Milieu granulaire, Milieu déformable, Méthode des Éléments Discrets, Dynamique des contacts non-réguliers, Méthode semi-régulière de Newton, Solveur itératif Gauss-Seidel non-linéaire, Méthode Primal-Dual Active Set, Écoulement granulaire, Paradigme de parallélisation, Couplage fluide-particules